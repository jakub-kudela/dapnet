%%%%%%%%%%%%%%%%%%%%%%%%%%%%%%%%%%%%%%%%%%%%%%%%%%%%%%%%%%%%%%%%%%%%%%%%%%%%%%%%
%% BAKALÁRSKA PRÁCA                                                           %%
%%                                                                            %%
%% Názov (sk): Algoritmy detekcie a korekcie lokálnych                        %%
%%             znehodnotení digitálneho audio signálu                         %%
%% Názov (en): Algorithms for Detection and Correction of Local               %%
%%             Degradations in Digital Audio Signals                          %%
%%                                                                            %%
%% Autor: Jakub Kúdela                                                        %%
%% Vedúci: Mgr. Daniel Toropila                                               %%
%%                                                                            %%
%% Akademický rok: 2011/2012                                                  %%
%%%%%%%%%%%%%%%%%%%%%%%%%%%%%%%%%%%%%%%%%%%%%%%%%%%%%%%%%%%%%%%%%%%%%%%%%%%%%%%%

\selectlanguage{english}
\nonfrenchspacing

\noindent
Title: Algorithms for Detection and Correction of Local Degradations in Digital Audio Signals\\
Author: Jakub Kúdela\\
Author's e-mail address: \url{jakub.kudela@gmail.com}\\
Department: Department of Theoretical Computer Science and Mathematical Logic\\
Thesis Supervisor: Mgr. Daniel Toropila\\
Supervisor's e-mail address: \url{daniel.toropila@mff.cuni.cz}\\

\noindent
Abstract: Local degradations in audio signal are discontinuities in their waveforms. They are caused by the nature of the recording process, or by aging of or damage to the recording medium. In many cases these discontinuities are unwanted while listening, and so there exists a number of methods, whose aim is to restore degraded recordings. In the introduction, this thesis informs the reader about selected algorithms for detection and correction of local degradations in digital audio signals. One of the discussed algorithms is a custom aplication of artificial neural networks to the given problem. The implementation of selected algorithms and experiments are both parts of the thesis. The goal of the experiments is to both objectively and subjectively compare the performances of the selected algorithms. The thesis proposes a method for the objective evaluation of the quality of detection and correction, which, as will be shown, largely corresponds to the subjective evaluation. Results of the experiments show that the custom application of neural networks---a non-linear model---is computationally intensive, and its results in limited time are not sufficient. The thesis shows that the best approach to clean records, given the common computing capabilities, is based on a linear autoregressive model.\\

\noindent
Keywords: digital audio signal restoration, multilayer perceptron neural network, autoregressive model

\selectlanguage{slovak}
\frenchspacing