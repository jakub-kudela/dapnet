%%%%%%%%%%%%%%%%%%%%%%%%%%%%%%%%%%%%%%%%%%%%%%%%%%%%%%%%%%%%%%%%%%%%%%%%%%%%%%%%
%% BAKALÁRSKA PRÁCA                                                           %%
%%                                                                            %%
%% Názov (sk): Algoritmy detekcie a korekcie lokálnych                        %%
%%             znehodnotení digitálneho audio signálu                         %%
%% Názov (en): Algorithms for Detection and Correction of Local               %%
%%             Degradations in Digital Audio Signals                          %%
%%                                                                            %%
%% Autor: Jakub Kúdela                                                        %%
%% Vedúci: Mgr. Daniel Toropila                                               %%
%%                                                                            %%
%% Akademický rok: 2011/2012                                                  %%
%%%%%%%%%%%%%%%%%%%%%%%%%%%%%%%%%%%%%%%%%%%%%%%%%%%%%%%%%%%%%%%%%%%%%%%%%%%%%%%%

\noindent
Názov: Algoritmy detekcie a korekcie lokálnych znehodnotení digitálneho audio signálu\\
Autor: Jakub Kúdela\\
E-mailová adresa autora: \url{jakub.kudela@gmail.com}\\
Katedra: Katedra teoretické informatiky a matematické logiky\\
Veducí práce: Mgr. Daniel Toropila\\
E-mailová adresa vedúceho: \url{daniel.toropila@mff.cuni.cz}\\

\noindent
Abstrakt: Lokálne znehodnotenia audio signálu sú nespojitosti v záznamovej stope. Sú zapričinené charakterom nahrávacieho procesu alebo stárnutím či poškodením záznamového média. V mnohých prípadoch sú tieto nespojitosti pri posluchu nežiadúce a tak existuje množstvo metód, ktoré si kladú za cieľ poškodené nahrávky reštaurovať. V úvode práca oboznámi čitateľa s vybranými algoritmami pre detekciu a korekciu lokálnych znehodnotení v digitálnych audio signáloch. Jeden z predostretých algoritmov v práci je vlastnou aplikáciou umelých neurónových sietí na danú problematiku. Súčasťou práce je implementácia vybraných algoritmov spolu s experimentami. Cieľom experimentov je objektívne aj subjektívne porovnať výkony vybraných algoritmov. V práci je navrhnutá metóda pre objektívne hodnotenie kvality detekcie a korekcie, ktorá, ako sa ukáže, do značnej miery odpovedá subjektívnemu hodnoteniu. Výsledky experimentov ukazujú, že vlastná aplikácia neurónových sietí, nelineárneho modelu, je výpočetne náročná a jej výsledky v obmedzenom čase nie sú postačujúce. V práci sa ukáže, že najvhodnejší prístup k čisteniu nahrávok v rámci bežných výpočetných možností je založený na lineárnom autoregresívnom modeli.\\

\noindent
Kľúčové slová: reštaurácia digitálneho audio signálu, eurónová sieť viacvrstvový perceptrón, autoregresívny model
